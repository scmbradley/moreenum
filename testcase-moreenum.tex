\documentclass[paper=A4,fontsize=12]{scrartcl}
\usepackage[T1]{fontenc}
\usepackage[utf8]{inputenc}
\usepackage{url}
\usepackage{etoolbox}
\usepackage{multicol}
\usepackage{listings}
\lstloadlanguages{[LaTeX]TeX} % I need to tell listings how I want code typeset. Here come some options:
\lstset{% 
    basicstyle=\ttfamily\small, 
    commentstyle=\itshape\ttfamily\small, 
    showspaces=false, 
    showstringspaces=false, 
    breaklines=true, 
    breakautoindent=true, 
    captionpos=t,
    frame=single,
    escapeinside={(*}{*)},
} 
\catcode`|=\active
\def|{\lstinline|}

\usepackage{moreenum}
\usepackage{etoolbox}
\author{Seamus Bradley}
\title{\texttt{moreenum}}
\usepackage{tgpagella}
\usepackage{tgheros}
%\usepackage{tgcursor}
\usepackage[ocr-a]{ocr}

\newcommand\testenumerate[1]{%
  \bigskip
  \begin{minipage}{0.4\linewidth}
  {\qquad\Large\texttt{\textbackslash #1}}
  \begin{enumerate}[label=\csname #1\endcsname*]
  \item Liberty
  \item Equality
  \item Fraternity\setcounter{enumi}{41}
  \item Meaning of life
  \end{enumerate}
  \end{minipage}
}
\newcommand\dotest[1]{%
  \let\do\testenumerate
  \docsvlist}

\begin{document}
\maketitle


\section{Basic procedure}

At the heart of each new enumeration is the following procedure:

\begin{lstlisting}
  \newcommand*{\macro}[1]{%
    \expandafter\@macro\csname c@#1\endcsname}
  \newcommand*{\@macro}[1]{%
    \translate{#1}}
  \AddEnumerateCounter{\macro}{\@macro}{distance}
\end{lstlisting}

From a user perspective, \lstinline+\macro+ takes a counter as its
argument and outputs, say, a
binary number or whatever you want. 
Actually, what it really does is turn a counter into a number and pass
the number to \lstinline+\@macro+ which does the real work.
It takes a number and translates it into the final representation.

Most of the cleverness is done by \lstinline+\translate+ and these are
mostly macros I've borrowed from other packages.

The |distance| is the widest entry in the enumeration.
|moreenum| hasn't been tested much with this parameter:
I've just guessed a bit at what's the widest enumerations are likely
to get.
Enumerations can \emph{theoretically} get up to 2147483647 items long.
Which would be rather a long number.\footnote{%
  |fmcount| doesn't seem to work with numbers that big, actually. 
  But even 131071 is \binarynum{131071}}

The |\greek| macro is a little more involved because it involves first
defining a macro that turns numbers into Greek letters.

\begin{lstlisting}
\newcommand*{\single@greek}[1]{%
  \expandafter\@single@greek\csname c@#1\endcsname
}
\newcommand*{\@single@greek}[1]{%
  $\ifcase#1\or\alpha\or\beta\or\gamma\or\delta\or\varepsilon
    \or\zeta\or\eta\or\theta\or\iota\or\kappa\or\lambda
    \or\mu\or\nu\or\xi\or o\or\pi\or\varrho\or\sigma
    \or\tau\or\upsilon\or\phi\or\chi\or\psi\or\omega
    \else\@ctrerr\fi$
}
\end{lstlisting}

Then you need to define what to do when you run out of letters.
You start again at $\alpha\alpha$.
The clever work there is done by the |alphalph| package.

\begin{lstlisting}
\newalphalph{\@greek}[alph]{\@single@greek}{24}
\newcommand*{\greek}[1]{%
  \expandafter\@greek\csname c@#1\endcsname
}

\AddEnumerateCounter{\greek}{\@greek}{$\omega$}
\end{lstlisting}

Some sophistication is required to get the |\translate|-style macros
to play nice with |\label| and |\ref| facilities.
This can be seen in the following example.

\begin{lstlisting}
\newcommand*{\enumHex}[1]{%
  \expandafter\@enumHex\csname c@#1\endcsname}
\newcommand*{\@enumHex}[1]{%
  \protect\Hexadecimalnum{\number#1}}
\AddEnumerateCounter{\enumHex}{\@enumHex}{AAAA}
\end{lstlisting}

The |\protect| makes sure the |\Hexadecimalnum| get written to the
|.aux| file, rather than expanded first.
The |\number| makes sure the number \emph{is} written to the |.aux|
file.\footnote{I'm actually guessing here. 
  I have no idea.
  I got the clue from |egreg| here:
  \url{http://tex.stackexchange.com/q/22234/215}}

\section{Limitations}

The biggest number TeX can handle is 2147483647.
I can't imagine this ever being a serious limitation to your
enumerating.

There are, however, some further limitations.
Certain |fmtcount| macros seem to break before they hit this
fundamental limit.
In brackets are the |moreenum|-defined enumerations affected.
\begin{itemize}
\item |\binary| and friends break at 131072 [|\enumbinary|]
\item |\hexadecimal| and friends break at 1048576 [|\enumhex| and |\enumHex|]
\item |\numberstring| and friends break at 100000 [|\nwords| and |\nthwords|]
\end{itemize}

None of these is a serious limitation.
If you desperately need bigger enumerations, they are fairly
straightforward to define yourself using |binhex| for the numbers and
|numname| for the words: these packages don't have these
limitations.\footnote{%
  Why don't I just use those packages instead?
  Because having |fmtcount| do most of the work means only loading one
  package instead of 3 (|numname|, |binhex| and |nth| or |engord|).}


\section{Examples of the enumerations}
Here are examples of all the kinds of enumeration that the package defines:
% \dotest{greek,Greek,enumHex,enumhex,enumbinary,raisenth,levelnth,nthwords,nwords}

\small\noindent
\testenumerate{greek}\hfil
\testenumerate{Greek}
\testenumerate{enumHex}\hfil
\testenumerate{enumhex}
\testenumerate{enumbinary}\hfil
\testenumerate{raisenth}
\testenumerate{levelnth}\hfil
\testenumerate{nthwords}
\testenumerate{nwords}


\end{document}

%%% Local Variables: 
%%% mode: latex
%%% TeX-master: t
%%% End: 
